\documentclass[12pt,a4paper]{article}

% ---------- Packages ----------
\usepackage[utf8]{inputenc}
\usepackage[T1]{fontenc}
\usepackage{graphicx}
\usepackage{amsmath, amssymb, amsfonts}
\usepackage{siunitx}
\usepackage{hyperref}
\usepackage{geometry}
\usepackage{caption}
\usepackage{subcaption}
\usepackage{listings}
\usepackage{xcolor}

\geometry{margin=1in}
\hypersetup{
    colorlinks=true,
    linkcolor=blue,
    citecolor=blue,
    urlcolor=blue,
}

% ---------- Code formatting ----------
\lstset{
    basicstyle=\ttfamily\footnotesize,
    breaklines=true,
    frame=single,
    backgroundcolor=\color{gray!5},
    numbers=left,
    numberstyle=\tiny,
    keywordstyle=\color{blue},
    commentstyle=\color{green!50!black},
    stringstyle=\color{orange},
    tabsize=4
}

% ---------- Document ----------
\begin{document}

\title{Smart Room Control System\\
\large Modeling, Simulation and Embedded Control using STM32 and Python}
\author{Author: \textbf{Juan Marin} \\ Institution: ASIMOV}
\date{\today}
\maketitle
 
\begin{abstract}
This document describes the design, modeling and implementation of a closed-room environmental control system. 
The physical plant (temperature, humidity and CO$_2$) is simulated in Python, while the control algorithm 
(Kalman Filter and Model Predictive Control) runs on an STM32F767ZI microcontroller. 
Both systems communicate through UART over USB. A graphical interface allows real-time visualization 
and modification of temperature, humidity and CO$_2$ setpoints.
\end{abstract}

\tableofcontents
\newpage

% ============================================================
\section{Introduction}
This project aims to design a smart indoor climate control system with a realistic yet computationally simple dynamic model. 
The goal is to maintain comfortable levels of temperature, humidity and CO$_2$ concentration using two actuators: 
an electric heater and a ventilation fan.

\subsection{System Overview}
The system consists of:
\begin{itemize}
    \item \textbf{Python Plant:} Simulates environmental dynamics and sensors.
    \item \textbf{STM32 Controller:} Runs estimation (Kalman Filter) and MPC control.
    \item \textbf{UART Communication:} Transfers measurements and control signals.
    \item \textbf{Graphical User Interface:} Displays real-time data and allows user interaction.
\end{itemize}

% ============================================================
\section{Physical Model and State Variables}
The room is modeled as a well-mixed air volume. 
The state vector is defined as:
\[
x = 
\begin{bmatrix}
T \\ w \\ c
\end{bmatrix}
\]
where:
\begin{itemize}
    \item $T$ [°C] — air temperature
    \item $w$ [kg/kg] — absolute humidity
    \item $c$ [ppm] — CO$_2$ concentration
\end{itemize}

The control inputs are:
\[
u =
\begin{bmatrix}
u_h \\ u_f
\end{bmatrix}
\]
where $u_h$ and $u_f$ represent heater and fan PWM duty cycles respectively.

Disturbances include outdoor air conditions and occupancy:
\[
d =
\begin{bmatrix}
T_o \\ w_o \\ c_o \\ N
\end{bmatrix}
\]

% ============================================================
\section{Mathematical Modeling}
\subsection{Continuous-Time Dynamics}
\begin{align}
\dot{T} &= \frac{q}{V}(T_o - T) + \frac{\eta_h P_h}{\rho c_p V}u_h + \frac{Q_{pers}}{\rho c_p V}N \\
\dot{w} &= \frac{q}{V}(w_o - w) + \frac{G_w}{\rho V}N \\
\dot{c} &= \frac{q}{V}(c_o - c) + \gamma_c N
\end{align}

where:
\begin{itemize}
    \item $q = q_{max} u_f + k_{stack}(T - T_o)$ is the total airflow (m³/s)
    \item $V$ is the room volume
    \item $\rho$ and $c_p$ are air density and heat capacity
    \item $\eta_h P_h$ is the effective heater power
    \item $Q_{pers}, G_w, \gamma_c$ represent heat, moisture and CO$_2$ generation per person
\end{itemize}

\subsection{Discretization}
Using forward Euler with sample time $T_s$:
\[
x_{k+1} = x_k + T_s \, f(x_k, u_k, d_k)
\]

\subsection{Linearization}
Around an operating point $(x^*, u^*, d^*)$:
\[
\delta x_{k+1} = A\delta x_k + B\delta u_k + E\delta d_k
\]
The matrices $A$, $B$, and $E$ are obtained from the Jacobians of $f(x,u,d)$.

% ============================================================
\section{Python Plant Implementation}
The plant simulation will be written in Python. It will:
\begin{itemize}
    \item Update the states using the discrete model.
    \item Add measurement noise to emulate real sensors.
    \item Send measurements to STM32 via UART.
    \item Receive control signals ($u_h$, $u_f$) and apply them.
\end{itemize}

Example Python structure:
\begin{lstlisting}[language=Python, caption={Basic structure of the plant model.}]
def q_flow(u_f, T, To):
    return q_max * u_f + k_stack * (T - To)

def f_cont(x, u, d):
    T, w, c = x
    uh, uf = u
    To, wo, co, N = d
    q = q_flow(uf, T, To)
    alpha = q / V
    dT = alpha*(To - T) + (eta_h*Ph)/(rho*cp*V)*uh + (Q_person)/(rho*cp*V)*N
    dw = alpha*(wo - w) + (G_w)/(rho*V)*N
    dc = alpha*(co - c) + gamma_c*N
    return np.array([dT, dw, dc])
\end{lstlisting}

% ============================================================
\section{Embedded Controller (STM32F767ZI)}
\subsection{Overview}
The STM32 firmware will:
\begin{enumerate}
    \item Receive sensor data via UART.
    \item Estimate states with a discrete Kalman Filter.
    \item Compute control actions using MPC.
    \item Send PWM values back to Python.
\end{enumerate}

\subsection{Communication Protocol (UART over USB)}
\begin{itemize}
    \item Baud rate: 115200 bps
    \item Format: JSON or comma-separated ASCII packets
    \item Direction:
    \begin{itemize}
        \item Python → STM32: $T, w, c, N$
        \item STM32 → Python: $u_h, u_f$
    \end{itemize}
\end{itemize}

Example packet:
\begin{verbatim}
<MEAS, 22.4, 0.0062, 750, 2>
<CTRL, 0.45, 0.70>
\end{verbatim}

% ============================================================
\section{Graphical User Interface}
The GUI (Python) will:
\begin{itemize}
    \item Display time-series plots for $T$, $w$, and $c$.
    \item Show heater and fan PWM signals.
    \item Allow user to set target values (setpoints).
    \item Start/stop the simulation.
\end{itemize}

Potential tools: \texttt{PyQt5}, \texttt{Tkinter}, or \texttt{Streamlit}.

% ============================================================
\section{Integration and Testing}
\begin{itemize}
    \item Verify UART data integrity.
    \item Validate the model by comparing simulated and real dynamics.
    \item Tune Kalman Filter noise matrices $Q$ and $R$.
    \item Tune MPC weights for comfort vs. energy trade-off.
\end{itemize}

% ============================================================
\section{Conclusions and Future Work}
This document establishes the foundation for implementing a closed-loop 
environmental control system using hybrid simulation and embedded hardware. 
Next steps include GUI integration, experimental validation, and possible 
extension to multi-zone or networked control systems.

% ============================================================
\bibliographystyle{ieeetr}
\bibliography{references}

\end{document}
